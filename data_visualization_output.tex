% This LaTeX was auto-generated from MATLAB code.
% To make changes, update the MATLAB code and export to LaTeX again.

\documentclass{article}

\usepackage[utf8]{inputenc}
\usepackage[T1]{fontenc}
\usepackage{lmodern}
\usepackage{graphicx}
\usepackage{color}
\usepackage{hyperref}
\usepackage{amsmath}
\usepackage{amsfonts}
\usepackage{epstopdf}
\usepackage[table]{xcolor}
\usepackage{matlab}
\usepackage[paperheight=795pt,paperwidth=614pt,top=72pt,bottom=72pt,right=72pt,left=72pt,heightrounded]{geometry}

\sloppy
\epstopdfsetup{outdir=./}
\graphicspath{ {./starproject_media/} }

\begin{document}

\begin{matlabcode}
% Loading the data:
load starData.mat

% Calculating the wavelenghts:
intervals = size(spectra, 1);
lambdaStart = 630.02;
lambdaDelta = 0.14;

lambdaEnd = lambdaStart + (intervals-1) * lambdaDelta;
lambda = (lambdaStart:lambdaDelta:lambdaEnd).';

% Redshift calculation for all stars
[sHa, idx] = min(spectra);
lambdaHa = lambda(idx);
z = (lambdaHa / 656.28) - 1;
speed = z * 299792.458;

% Loop through each star for analysis and visualization
for i = 1:size(spectra, 2) % Looping through each star
    spectrum = spectra(:, i); % Extracting the spectrum of the current star
    wavelength = lambda; % Wavelength values

    % Basic spectral analysis (finding the peak value)
    [peakIntensity, peakIndex] = max(spectrum);
    peakWavelength = wavelength(peakIndex);

    % Movement analysis
    restWavelength = 656.28; % Hydrogen alpha line in nanometers
    observedWavelength = peakWavelength;
    z = (observedWavelength - restWavelength) / restWavelength;

    % Determining the direction of movement, star name and number
    starName = starnames{i};
    if z > 0
        disp([starName, ' (Star ', num2str(i), ') is moving away.']);
    else
        disp([starName, ' (Star ', num2str(i), ') is moving towards.']);
    end

    % Visualization
    figure;
    plot(wavelength, spectrum);
    hold on;
    plot(peakWavelength, peakIntensity, 'r*', 'MarkerSize', 10);
    title(['Spectrum of ', starName, ' (Star ', num2str(i), ')']);
    xlabel('Wavelength');
    ylabel('Intensity');
    hold off;

    % Add some blank line after each plot for spacing
    disp(' '); 
    disp(' ');
    disp(' ');
end
\end{matlabcode}
\begin{matlaboutput}
HD  30584 (Star 1) is moving towards.
\end{matlaboutput}
\begin{center}
\includegraphics[width=\maxwidth{57.90265930757652em}]{figure_0.eps}
\end{center}
\begin{matlaboutput}
 
 
 
HD  10032 (Star 2) is moving towards.
\end{matlaboutput}
\begin{center}
\includegraphics[width=\maxwidth{57.90265930757652em}]{figure_1.eps}
\end{center}
\begin{matlaboutput}
 
 
 
HD  64191 (Star 3) is moving towards.
\end{matlaboutput}
\begin{center}
\includegraphics[width=\maxwidth{57.90265930757652em}]{figure_2.eps}
\end{center}
\begin{matlaboutput}
 
 
 
HD   5211 (Star 4) is moving towards.
\end{matlaboutput}
\begin{center}
\includegraphics[width=\maxwidth{57.90265930757652em}]{figure_3.eps}
\end{center}
\begin{matlaboutput}
 
 
 
HD  56030 (Star 5) is moving towards.
\end{matlaboutput}
\begin{center}
\includegraphics[width=\maxwidth{57.90265930757652em}]{figure_4.eps}
\end{center}
\begin{matlaboutput}
 
 
 
HD  94028 (Star 6) is moving towards.
\end{matlaboutput}
\begin{center}
\includegraphics[width=\maxwidth{57.90265930757652em}]{figure_5.eps}
\end{center}
\begin{matlaboutput}
 
 
 
SAO102986 (Star 7) is moving towards.
\end{matlaboutput}
\begin{center}
\includegraphics[width=\maxwidth{57.90265930757652em}]{figure_6.eps}
\end{center}
\begin{matlaboutput}
 
 
 
\end{matlaboutput}
\begin{matlabcode}


disp("Stars' wavelenghts comparison:")
\end{matlabcode}
\begin{matlaboutput}
Stars' wavelenghts comparison:
\end{matlaboutput}
\begin{matlabcode}
plot(lambda,spectra)
legend(starnames)
\end{matlabcode}
\begin{center}
\includegraphics[width=\maxwidth{57.90265930757652em}]{figure_7.eps}
\end{center}
\begin{matlabcode}



% Identifying stars moving away/toward
moveAway = starnames(speed > 0);
moveToward = starnames(speed <=0);
disp("These stars are moving away:")
\end{matlabcode}
\begin{matlaboutput}
These stars are moving away:
\end{matlaboutput}
\begin{matlabcode}
disp(moveAway)
\end{matlabcode}
\begin{matlaboutput}
    "HD   5211"
    "HD  56030"
    "HD  94028"
\end{matlaboutput}
\begin{matlabcode}

disp("and these stars are moving toward the Earth:") 
\end{matlabcode}
\begin{matlaboutput}
and these stars are moving toward the Earth:
\end{matlaboutput}
\begin{matlabcode}
disp(moveToward)
\end{matlabcode}
\begin{matlaboutput}
    "HD  30584"
    "HD  10032"
    "HD  64191"
    "SAO102986"
\end{matlaboutput}

\end{document}
